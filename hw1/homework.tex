\documentclass[fontset=windows]{article}

\usepackage{fancyhdr}
\usepackage{extramarks}
\usepackage{amsmath}
\usepackage{amsthm}
\usepackage{amsfonts}
\usepackage{tikz}
\usepackage[plain]{algorithm}
\usepackage{algpseudocode}
\usepackage{ctex}

\usetikzlibrary{automata,positioning}

%
% Basic Document Settings
%

\topmargin=-0.45in
\evensidemargin=0in
\oddsidemargin=0in
\textwidth=6.5in
\textheight=9.0in
\headsep=0.25in

\linespread{1.1}

\pagestyle{fancy}
\lhead{\hmwkAuthorName}
\chead{\hmwkClass: \hmwkTitle}
\rhead{\firstxmark}
\lfoot{\lastxmark}
\cfoot{\thepage}

\renewcommand\headrulewidth{0.4pt}
\renewcommand\footrulewidth{0.4pt}

\setlength\parindent{0pt}

%
% Create Problem Sections
%

\newcommand{\enterProblemHeader}[1]{
    \nobreak\extramarks{}{Problem \arabic{#1} continued on next page\ldots}\nobreak{}
    \nobreak\extramarks{Problem \arabic{#1} (continued)}{Problem \arabic{#1} continued on next page\ldots}\nobreak{}
}

\newcommand{\exitProblemHeader}[1]{
    \nobreak\extramarks{Problem \arabic{#1} (continued)}{Problem \arabic{#1} continued on next page\ldots}\nobreak{}
    \stepcounter{#1}
    \nobreak\extramarks{Problem \arabic{#1}}{}\nobreak{}
}

\setcounter{secnumdepth}{0}
\newcounter{partCounter}
\newcounter{homeworkProblemCounter}
\setcounter{homeworkProblemCounter}{1}
\nobreak\extramarks{Problem \arabic{homeworkProblemCounter}}{}\nobreak{}

%
% Homework Problem Environment
%
% This environment takes an optional argument. When given, it will adjust the
% problem counter. This is useful for when the problems given for your
% assignment aren't sequential. See the last 3 problems of this template for an
% example.
%
\newenvironment{homeworkProblem}[1][-1]{
    \ifnum#1>0
        \setcounter{homeworkProblemCounter}{#1}
    \fi
    \section{Problem \arabic{homeworkProblemCounter}}
    \setcounter{partCounter}{1}
    \enterProblemHeader{homeworkProblemCounter}
}{
    \exitProblemHeader{homeworkProblemCounter}
}

%
% Homework Details
%   - Title
%   - Due date
%   - Class
%   - Section/Time
%   - Instructor
%   - Author
%

\newcommand{\hmwkTitle}{Homework 1}
\newcommand{\hmwkDueDate}{March-09, 2022}
\newcommand{\hmwkClass}{Concrete Math}
\newcommand{\hmwkClassInstructor}{Professor Chen Xue}
\newcommand{\hmwkStuNum}{\textbf{SA21011018}}
\newcommand{\hmwkAuthorName}{\textbf{Zhou Enshuai}}

%
% Title Page
%

\title{
    \vspace{2in}
    \textmd{\textbf{\hmwkClass:\ \hmwkTitle}}\\
    \normalsize\vspace{0.1in}\small{Due\ on\ \hmwkDueDate\ at 14:00}\\
    \vspace{0.1in}\large{\textit{\hmwkClassInstructor}}
    \vspace{3in}
}

\author{\hmwkStuNum\\ \hmwkAuthorName}
\date{\today}

\renewcommand{\part}[1]{\textbf{\large Part \Alph{partCounter}}\stepcounter{partCounter}\\}

%
% Various Helper Commands
%

% Useful for algorithms
\newcommand{\alg}[1]{\textsc{\bfseries \footnotesize #1}}

% For derivatives
\newcommand{\deriv}[1]{\frac{\mathrm{d}}{\mathrm{d}x} (#1)}

% For partial derivatives
\newcommand{\pderiv}[2]{\frac{\partial}{\partial #1} (#2)}

% Integral dx
\newcommand{\dx}{\mathrm{d}x}

% Alias for the Solution section header
\newcommand{\solution}{\textbf{\large Solution}}

% Probability commands: Expectation, Variance, Covariance, Bias
\newcommand{\E}{\mathrm{E}}
\newcommand{\Var}{\mathrm{Var}}
\newcommand{\Cov}{\mathrm{Cov}}
\newcommand{\Bias}{\mathrm{Bias}}

\begin{document}

\maketitle

\pagebreak

\begin{homeworkProblem}
    Let $z_1 < z_2 < ... < z_n$ be the correct order of all elements in the array A. Then
    consider those pivots chosen in the natural order of QuickSort. For any $z_j$ and $z_k$, argue that

    \begin{enumerate}
        \item If the 1st pivot chosen among $z_j,... ,z_k$ is not $z_j$ or $z_k$, the algorithm won't compare $z_j$ and $z_k$.
        \item $z_j$ or $z_k$ get compared only if the 1st pivot chosen among  $z_j,... ,z_k$ is either $z_j$ or $z_k$.
    \end{enumerate}

    \solution

    不妨设$j<k$,则$z_j<z_k$。

    \begin{enumerate}
        \item 设第一个选自$z_j,... ,z_k$的pivot是$z_i$,且$z_i\neq z_j,z_i\neq z_k$,那么$z_j<z_i<z_k$,
              这时区间$z_j,... ,z_k$会以$z_i$为pivot划分成两个子区间,$z_j$在前一子区间,$z_k$在后一子区间。在此之后,$z_j$ 和 $z_k$不在同一子区间内,自然不会比较;
              而在此之前,$z_j$ 和 $z_k$也没有成为pivot(否则$z_i$不是第一个pivot),自然也不会比较。所以,$z_j$ 和 $z_k$不会比较。
        \item 即证明:“$z_j$ 和 $z_k$发生比较”$\Leftrightarrow$“第一个选自$z_j,... ,z_k$的pivot是$z_j$ 或 $z_k$”。
              先证明充分性:由1的逆否命题可知,若$z_j$ 和 $z_k$发生比较,则第一个选自$z_j,... ,z_k$的pivot是$z_j$ 或 $z_k$;
              再证明必要性:若第一个选自$z_j,... ,z_k$的pivot是$z_j$ 或 $z_k$,不妨设选的pivot是$z_j$,则说明之前没有选自该区间的pivot,那么该区间还未被划分。
              此时,该区间内的每一个元素都要和pivot\ $z_j$比较,$z_k$也不例外,所以$z_j$ 和 $z_k$发生比较。
    \end{enumerate}

\end{homeworkProblem}

\pagebreak

\begin{homeworkProblem}

    在一块厚奶酪上划出五道直的切痕,可以得到多少块奶酪?(在你划切痕时,奶酪必须保持在它原来的位置上,且每道切痕必定与三维空间中的一个平面相对应。)
    对$P_n$求一个递归关系,这里$P_n$表示n个不同的平面所能定义的三维区域的最大个数。

    \solution

    切第n刀时,第n刀平面与之前的n-1个平面相交,这n-1个平面在第n个平面上产生n-1条交线,这n-1条交线将第n个平面划分成$L_{n-1}$个子平面。
    对第n刀而言,增加的奶酪块是从原有的奶酪块区域分裂出来的,第n刀平面上每个子平面和原有的1个奶酪块区域相交,从而将该区域分裂成了2个子区域,也即增加了一个奶酪块区域。
    所以第n刀平面上子平面的个数就是切第n刀时新增的奶酪块数。由此可得:
    $$P_n=P_{n-1}+L_{n-1}$$
    由之前二维平面分割数$L_n$的递归关系可得:
    $$L_n=\frac{n(n+1)}{2}+1$$
    则$P_n$的递归关系为:
    $$
        P_n=
        \left\{
        \begin{aligned}
             & P_{n-1}+\frac{n(n-1)}{2}+1\  & (n\ge2) \\
             & 2\                           & (n=1)
        \end{aligned}
        \right.
    $$

    当$n\ge2$时,可以累加求和:
    $$
        \begin{aligned}
            P_n & = P_1+\sum_{i=2}^{n} L_{i-1}                                        \\
                & =2+\sum_{i=2}^{n} (\frac{i(i-1)}{2}+1)                              \\
                & =n+1+\sum_{i=2}^{n} \frac{i(i-1)}{2}                                \\
                & =n+1+\sum_{i=1}^{n-1} \frac{i(i+1)}{2}                              \\
                & =n+1+\frac{1}{2} \sum_{i=1}^{n-1}i^2 +\frac{1}{2} \sum_{i=1}^{n-1}i \\
                & =n+1+\frac{n(n-1)(2n-1)}{12} +\frac{n(n-1)}{4}                      \\
        \end{aligned}
    $$

    可以发现$n=1$时,$P_n$也满足上述公式,所以综上:
    $$
        P_n = n+1+\frac{n(n-1)(2n-1)}{12} +\frac{n(n-1)}{4}
    $$

\end{homeworkProblem}

\pagebreak

\begin{homeworkProblem}
    约瑟夫有一个朋友,他站在倒数第二的位置上因而获救。当每隔一个人就有一人被处死时,倒数第二个幸存者的号码$I(n)$是多少?

    \solution

    和求$J(n)$的思路一样,分成奇数和偶数两种情况讨论:
    $$
        \begin{aligned}
             & I(2)    = 2              \\
             & I(3)    = 1              \\
             & I(2n)   =2I(n)-1 & (n>1) \\
             & I(2n+1) =2I(n)+1 & (n>1)
        \end{aligned}
    $$
    当n为2的幂次时,即$n=2^k$,$I(2n)-1=2(I(n)-1)$,代入可得:
    $$
        \begin{aligned}
            I(2^k)-1 & = 2^1(I(2^{k-1})-1) \\
                     & = 2^2(I(2^{k-2})-1) \\
                     & ...                 \\
                     & = 2^{k-1}(I(2)-1)   \\
                     & = 2^{k-1}
        \end{aligned}
    $$
    也就是说,$I(2^k)=2^{k-1}+1$。

    回归本题,讨论一般情况下的n,设$n=2^m+l\ (0\le l<2^m)$,在删去$l$个人之后,剩下$2^m$个人,
    将这$2^m$个人重新编号为$1,2,3,4,...,2^m$,原编号与新编号映射关系为:

    \begin{center}
        \begin{tabular}{cccccccc}
            $\color{blue}1$       & $\color{blue}3$       & $\color{blue}...$ & $\color{blue}2l-1$ & $\color{green}2l+1$ & $\color{green}2l+2$ & $\color{green}...$ & $\color{green}2^m+l$ \\
            $\downarrow$          & $\downarrow$          & $\downarrow$      & $\downarrow$       & $\downarrow$        & $\downarrow$        & $\downarrow$       & $\downarrow$         \\
            $\color{blue}2^m-l+1$ & $\color{blue}2^m-l+2$ & $\color{blue}...$ & $\color{blue}2^m$  & $\color{green}1$    & $\color{green}2$    & $\color{green}...$ & $\color{green}2^m-l$ \\
        \end{tabular}
    \end{center}

    在$1,2,...,2^m$新编号中,被删去的倒数第二个幸存者新编号为$I(2^m)=2^{m-1}+1$,根据该新编号和映射表可以得到:
    \begin{itemize}
        \item 当$2^{m-1}+1\le2^m-l$,即$l<2^{m-1}$时,$2^{m-1}+1$对应的原编号为$2^{m-1}+1+2l$,即$I(n)=2l+1+2^{m-1}$。
        \item 当$2^{m-1}+1>2^m-l$,即$l\ge2^{m-1}$时,$2^{m-1}+1$对应的原编号为$2(2^{m-1}+1-(2^m-l))-1$,即$I(n)=2l+1-2^{m}$。
    \end{itemize}
    综上:
    $$
        I(n)=
        \left\{
        \begin{aligned}
             & 2l+1+2^{m-1} = J(n)+2^{m-1} & 0\le l<2^{m-1}    \\
             & 2l+1-2^{m}  =  J(n)-2^{m}   & 2^{m-1}\le l <2^m
        \end{aligned}
        \right.
    $$
    其中$n=2^m+l\ (0\le l<2^m)$。
\end{homeworkProblem}

\pagebreak

\begin{homeworkProblem}
    利用求和因子来求解递归式:
    $$
        \begin{aligned}
             & T_0=5;                              \\
             & 2T_n=nT_{n-1}+3\times n!,\quad n>0.
        \end{aligned}
    $$

    \solution

    $$
        a_n = 2,\ b_n=n,\ c_n=3n! \\
    $$
    求和因子设为$s_n$:
    $$
        \frac{s_n}{s_{n-1}} = \frac{a_{n-1}}{b_n}
    $$
    由于$s_1$最后会消去,所以不妨设$s_1=1$,则:
    $$
        s_n = \frac{a_{n-1}a_{n-2}...a_1}{b_n b_{n-1}... b_2}=\frac{2^{n-1}}{n!}
    $$
    代入$s_n$可得:
    $$
        \begin{aligned}
            T_n & = \frac{s_1 b_1 T_0 + \sum_{k=1}^{n} s_k c_k}{s_n a_n}    \\
                & =\frac{5 + \sum_{k=1}^{n} 3\times2^{k-1}}{\frac{2^n}{n!}} \\
                & =\frac{3\times2^n+2}{\frac{2^n}{n!}}                      \\
                & =n!\cdot(3+2^{1-n})
        \end{aligned}
    $$
\end{homeworkProblem}

\pagebreak

\begin{homeworkProblem}
    证明拉格朗日恒等式:
    $$
        \sum_{1 \le j<k \le n}\left(a_{j} b_{k}-a_{k} b_{j}\right)^{2}=\left(\sum_{k=1}^{n} a_{k}^{2}\right)\left(\sum_{k=1}^{n} b_{k}^{2}\right)-\left(\sum_{k=1}^{n} a_{k} b_{k}\right)^{2}
    $$

    \solution

    可以交换求和式中的$k,j$变量名得到:
    $$
        \begin{aligned}
            \sum_{1 \le j<k \le n}(a_{j} b_{k}-a_{k} b_{j})^2=\sum_{1 \le j<k \le n}(a_{k} b_{j}-a_{j} b_{k})^2=\sum_{1 \le k<j \le n}(a_{j} b_{k}-a_{k} b_{j})^2
        \end{aligned}
    $$
    所以:
    $$
        \begin{aligned}
            \sum_{1 \le j,k \le n}(a_{j} b_{k}-a_{k} b_{j})^2 & = \sum_{1 \le j<k \le n}(a_{j} b_{k}-a_{k} b_{j})^2+\sum_{1 \le k<j \le n}(a_{j} b_{k}-a_{k} b_{j})^2 + \sum_{1 \le j=k \le n}(a_{j} b_{k}-a_{k} b_{j})^2 \\
                                                              & = 2\sum_{1 \le j<k \le n}(a_{j} b_{k}-a_{k} b_{j})^2 + \sum_{1 \le j \le n}(a_{j} b_{j}-a_{j} b_{j})^2                                                    \\
                                                              & = 2\sum_{1 \le j<k \le n}(a_{j} b_{k}-a_{k} b_{j})^2
        \end{aligned}
    $$
    即:
    $$
        \begin{aligned}
            \sum_{1 \le j<k \le n}(a_{j} b_{k}-a_{k} b_{j})^2 & = \frac{1}{2}\sum_{1 \le j,k \le n}(a_{j} b_{k}-a_{k} b_{j})^2                                                                                           \\
                                                              & = \frac{1}{2}\sum_{1 \le j,k \le n}(a_j^2 b_k^2 + a_k^2 b_j^2 - 2 a_j b_j a_k b_k)                                                                       \\
                                                              & = \frac{1}{2}\left[\sum_{1 \le j,k \le n} a_j^2 b_k^2 + \sum_{1 \le j,k \le n} a_k^2 b_j^2 - 2 \sum_{1 \le j,k \le n} a_j b_j a_k b_k\right]                        \\
                                                              & = \frac{1}{2}\left[\left(\sum_{k=1}^n a_k^2\right) \left(\sum_{k=1}^n b_k^2\right) + \left(\sum_{k=1}^n a_k^2\right) \left(\sum_{k=1}^n b_k^2\right)  - 2\left(\sum_{k=1}^n  a_k b_k\right) \left(\sum_{k=1}^n  a_k b_k\right)\right] \\
                                                              & = \left(\sum_{k=1}^n a_k^2\right) \left(\sum_{k=1}^n b_k^2\right) - \left(\sum_{k=1}^n  a_k b_k\right) ^2
        \end{aligned}
    $$
    证毕。


\end{homeworkProblem}

\end{document}p
