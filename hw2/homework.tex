\documentclass[fontset=windows]{article}

\usepackage{fancyhdr}
\usepackage{extramarks}
\usepackage{amsmath}
\usepackage{amsthm}
\usepackage{amsfonts}
\usepackage{tikz}
\usepackage{algorithm}
\usepackage{algpseudocode}
\usepackage{ctex}
\usepackage{manfnt}

\renewcommand{\algorithmicrequire}{\textbf{Input:}}  % Use Input in the format of Algorithm
\renewcommand{\algorithmicensure}{\textbf{Output:}} % Use Output in the format of Algorithm

\usetikzlibrary{automata,positioning}

%
% Basic Document Settings
%

\topmargin=-0.45in
\evensidemargin=0in
\oddsidemargin=0in
\textwidth=6.5in
\textheight=9.0in
\headsep=0.25in

\linespread{1.1}

\pagestyle{fancy}
\lhead{\hmwkAuthorName}
\chead{\hmwkClass: \hmwkTitle}
\rhead{\firstxmark}
\lfoot{\lastxmark}
\cfoot{\thepage}

\renewcommand\headrulewidth{0.4pt}
\renewcommand\footrulewidth{0.4pt}

\setlength\parindent{0pt}

%
% Create Problem Sections
%

\newcommand{\enterProblemHeader}[1]{
    \nobreak\extramarks{}{Problem \arabic{#1} continued on next page\ldots}\nobreak{}
    \nobreak\extramarks{Problem \arabic{#1} (continued)}{Problem \arabic{#1} continued on next page\ldots}\nobreak{}
}

\newcommand{\exitProblemHeader}[1]{
    \nobreak\extramarks{Problem \arabic{#1} (continued)}{Problem \arabic{#1} continued on next page\ldots}\nobreak{}
    \stepcounter{#1}
    \nobreak\extramarks{Problem \arabic{#1}}{}\nobreak{}
}

\setcounter{secnumdepth}{0}
\newcounter{partCounter}
\newcounter{homeworkProblemCounter}
\setcounter{homeworkProblemCounter}{1}
\nobreak\extramarks{Problem \arabic{homeworkProblemCounter}}{}\nobreak{}

%
% Homework Problem Environment
%
% This environment takes an optional argument. When given, it will adjust the
% problem counter. This is useful for when the problems given for your
% assignment aren't sequential. See the last 3 problems of this template for an
% example.
%
\newenvironment{homeworkProblem}[1][-1]{
    \ifnum#1>0
        \setcounter{homeworkProblemCounter}{#1}
    \fi
    \section{Problem \arabic{homeworkProblemCounter}}
    \setcounter{partCounter}{1}
    \enterProblemHeader{homeworkProblemCounter}
}{
    \exitProblemHeader{homeworkProblemCounter}
}

%
% Homework Details
%   - Title
%   - Due date
%   - Class
%   - Section/Time
%   - Instructor
%   - Author
%

\newcommand{\hmwkTitle}{Homework 2}
\newcommand{\hmwkDueDate}{March-23, 2022}
\newcommand{\hmwkClass}{Concrete Math}
\newcommand{\hmwkClassInstructor}{Professor Chen Xue}
\newcommand{\hmwkStuNum}{\textbf{SA21011018}}
\newcommand{\hmwkAuthorName}{\textbf{Zhou Enshuai}}

%
% Title Page
%

\title{
    \vspace{2in}
    \textmd{\textbf{\hmwkClass:\ \hmwkTitle}}\\
    \normalsize\vspace{0.1in}\small{Due\ on\ \hmwkDueDate\ at 14:00}\\
    \vspace{0.1in}\large{\textit{\hmwkClassInstructor}}
    \vspace{3in}
}

\author{\hmwkStuNum\\ \hmwkAuthorName}
\date{\today}

\renewcommand{\part}[1]{\textbf{\large Part \Alph{partCounter}}\stepcounter{partCounter}\\}

%
% Various Helper Commands
%

% Useful for algorithms
\newcommand{\alg}[1]{\textsc{\bfseries \footnotesize #1}}

% For derivatives
\newcommand{\deriv}[1]{\frac{\mathrm{d}}{\mathrm{d}x} (#1)}

% For partial derivatives
\newcommand{\pderiv}[2]{\frac{\partial}{\partial #1} (#2)}

% Integral dx
\newcommand{\dx}{\mathrm{d}x}

% Alias for the Solution section header
\newcommand{\solution}{\textbf{\large Solution}}

% Probability commands: Expectation, Variance, Covariance, Bias
\newcommand{\E}{\mathrm{E}}
\newcommand{\Var}{\mathrm{Var}}
\newcommand{\Cov}{\mathrm{Cov}}
\newcommand{\Bias}{\mathrm{Bias}}

\begin{document}

\maketitle

\pagebreak

\begin{homeworkProblem}
    Modify QuickSort algorithm to find the $k^{th}$ largest number: Given k and a
    sequence of n numbers $a1,...,an$ (not sorted), output the $k^{th}$ largest number.

    \begin{enumerate}
        \item Show the pseudo-code of your algorithm.
        \item Prove its expected running time is $O(n)$ by the probability argument about comparing pairs of elements.
    \end{enumerate}

    \solution

    \textbf{1.}

    \begin{algorithm}[h]
        \caption{Find the $k^{th}$ largest number in an array}
        \label{alg::find_kth}
        \begin{algorithmic}[1]
            \Require
            $A$: a sequence of n numbers;
            $begin$: the index of the first number in $A$;
            $end$: the index of the last number in $A$;
            $k$: the ordinal number.
            \Ensure
            the $k^{th}$ largest number.
            \Function {Find-Kth}{A, begin, end, k}
            \If{$k > (begin-end+1) $}
            \State \Return{}
            \EndIf
            \State index = \Call{Partition}{A, begin, end}
            \If{end - k + 1 < index}
            \State \Call{Find-Kth}{A, start, index - 1, k-(end - index + 1)}
            \ElsIf{end - k + 1 > index}
            \State \Call{Find-Kth}{A, index +1, end, k}
            \Else
            \State \Return{A[index]}
            \EndIf
            \EndFunction
            \\
            \Function{Partition}{A, p, r}
            \State x = A[r]
            \State i = p-1
            \For{ j = p to r-1}
            \If{A[j] <= x}
            \State i = i+1
            \State swap(A[i], A[j])
            \EndIf
            \EndFor
            \State swap(A[i+1], A[r])
            \State \Return{i+1}
            \EndFunction
        \end{algorithmic}
    \end{algorithm}

    \textbf{2.}

    设为期望时间为$T(n)$,已知调用Partition时间为$O(n)$,最大比较次数为$n$。设枢轴变量是第l大的元素,则$p(l=i)=\frac{1}{n},\ i=1,2,...,n$,
    若$l<k$,则搜索区间缩小为$[l+1,n]$,若$l>k$,则搜索区间缩小为$[1,l-1]$,若$l=k$,则搜索结束。
    则:
    $$
        \begin{aligned}
            T(n) & \le \sum_{l=1}^{n}\frac{1}{n}\left[[l<k]\times T(n-l) +[l>k]\times T(l-1)\right]                                                      + n+1 \\
                 & \le \sum_{l=1}^{n}\frac{1}{n}\left[[l<k]\times T(n-l) +[l>k]\times T(l-1) + [l==k]\times T(l-1)\right]                             + n+1    \\
                 & \le \sum_{l=1}^{n}\frac{1}{n}\left[[l<k] T(max(n-l,l-1)) +[l>k] T(max(n-l,l-1)) + [l==k] T(max(n-l,l-1))\right]  + n+1                      \\
                 & \le \frac{1}{n}\sum_{l=1}^{n}T(max(n-l,l-1))                                                                                       + n+1    \\
                 & \le \frac{1}{n}[\sum_{l=1}^{\lfloor \frac{n+1}{2} \rfloor}T(max(n-l,l-1)) + \sum_{l=\lceil \frac{n+1}{2} \rceil}^{n}T(max(n-l,l-1))]+ n+1   \\
                 & \le \frac{1}{n}[\sum_{l=1}^{\lfloor \frac{n+1}{2} \rfloor}T(n-l) + \sum_{l=\lceil \frac{n+1}{2} \rceil}^{n}T(l-1)]+ n+1                     \\
                 & \le \frac{1}{n}[\sum_{l=\lfloor \frac{n}{2} \rfloor}^{n-1}T(l) + \sum_{l=\lceil \frac{n-1}{2} \rceil}^{n-1}T(l)]+ n+1                       \\
                 & \le \frac{2}{n}\sum_{l=\lfloor \frac{n}{2} \rfloor}^{n-1}T(l) + n+1                                                                         \\
        \end{aligned}
    $$

    可得:
    $$
        \begin{aligned}
            nT(n) & \le 2\sum_{l=\lfloor \frac{n}{2} \rfloor}^{n-1}T(l) + n^2+n \\                                                                        \\
        \end{aligned}
    $$
    由于求的是性能上界,不妨设等号成立:
    $$
        \begin{aligned}
            nT(n)       & = 2\sum_{l=\lfloor \frac{n}{2} \rfloor}^{n-1}T(l) + n^2+n         \\
            (n-1)T(n-1) & = 2\sum_{l=\lfloor \frac{n-1}{2} \rfloor}^{n-2}T(l) + (n-1)^2+n-1 \\
        \end{aligned}
    $$
    两式相减可得:
    $$
        \begin{aligned}
             & nT(n) -(n-1)T(n-1)  = T(n-1) + 2n + T(\frac{n}{2}-1)[n\%2==0] \\
             & nT(n) -(n-1)T(n-1)  \le T(n-1) + 2n                           \\
             & nT(n)               \le nT(n-1) + 2n                          \\
             & T(n)                \le T(n-1) + 2                            \\
             & T(n)                \le 2n                                    \\
        \end{aligned}
    $$

    所以,$T(n)=O(n)$

\end{homeworkProblem}

\pagebreak

\begin{homeworkProblem}

    Prove that Euclid’s algorithm to compute $gcd(m, n)$ runs in time $O(log m+ log n)$
    assuming all integer-operations are done in $O(1)$ time.

    \solution

    设执行时间为$T(m,n)$,显然$T(m,n)$是关于$m,n$单调递增的。
    若$m=n$,则复杂度为$O(1)$,结论是平凡的;
    若$n=1$,则复杂度为$T(m,1)=O(1)$,结论也是平凡的;
    若$m\neq n$,不妨设$m>n$,可以分两种情况:
    \begin{itemize}
        \item 若$m\ge 2n$,则$m \mod n < n \le \frac{m}{2}$
        \item 若$m< 2n$,则$m \mod n =m-n <m- \frac{m}{2}=\frac{m}{2}$
    \end{itemize}
    所以综上$m\mod n\le \frac{m}{2}$,
    则:
    $$
        \begin{aligned}
            T(m,n) & = 1+ T(n, m \mod n)                    \\
                   & = 1+1 + T(m \mod n, n \mod (m \mod n)) \\
                   & \le 2 + T(\frac{m}{2}, \frac{n}{2})    \\
                   & \le 2 \log m+2\log n +O(1)             \\
                   & = O(\log m+\log n)
        \end{aligned}
    $$
    综上$T(m,n)=O(\log m+\log n)$

\end{homeworkProblem}

\pagebreak

\begin{homeworkProblem}
    \begin{enumerate}
        \item 将$\sum_{k=1}^n k2^k$重新改写成多重和式$\sum_{1\le j\le k\le n}2^k$的形式来对它进行计算
        \item 用正文中的方法5来计算$\mbox{\mancube}_n=\sum_{k=1}^nk^3$:首先记$\mbox{\mancube}_n+\square_n=2\sum_{1\le j\le k\le n}jk$然后应用(2.33)
    \end{enumerate}

    \solution

    \textbf{1.}

    $$
        \begin{aligned}
            \sum_{k=1}^n k2^k & = \sum_{1\le j\le k\le n}2^k    \\
                              & = \sum_{j=1}^n \sum_{k=j}^n 2^k \\
                              & = \sum_{j=1}^n (2^{n+1}-2^j)    \\
                              & =(n -1)2^{n+1}   +2             \\
        \end{aligned}
    $$

    \textbf{2.}

    $$
        \begin{aligned}
            \mbox{\mancube}_n+\square_n & =\sum_{k=1}^n(k^3+k^2)                   \\
                                        & =\sum_{k=1}^n(\frac{k^2+k}{2}\times 2k)  \\
                                        & =\sum_{k=1}^n(\frac{k(k+1)}{2}\times 2k) \\
                                        & =2\sum_{k=1}^n(\sum_{j=1}^k j\times k)   \\
                                        & = 2\sum_{1\le j\le k\le n}jk             \\
                                        & =(\sum_{k=1}^n k)^2+\sum_{k=1}^n k^2     \\
                                        & =(\sum_{k=1}^n k)^2+\square_n            \\
        \end{aligned}
    $$
    两边约去$\square_n$,可得:
    $$
        \mbox{\mancube}_n=(\sum_{k=1}^n k)^2=\frac{n^2(n+1)^2}{4}
    $$


\end{homeworkProblem}


\pagebreak


\begin{homeworkProblem}
    \begin{enumerate}
        \item 证明,表达式$\left\lceil\frac{2 x+1}{2}\right\rceil-\left\lceil\frac{2 x+1}{4}\right\rceil+\left\lfloor\frac{2 x+1}{4}\right\rfloor$ 总是等于$\lfloor x \rfloor$或者$\lceil x \rceil$
        \item 证明序列$1, 2, 2,3,3,3, 4, 4, 4, 4,5,5,5,5,5,$的第n个元素是$\left\lfloor\sqrt{2n}+\frac{1}{2}\right\rfloor$
    \end{enumerate}

    \solution

    \textbf{1.}

    对于任意整数n显然有:
    $$
        \lceil n \rceil - \lfloor n \rfloor = 0
    $$
    对于任意非整数y显然有:
    $$
        \lceil y \rceil - \lfloor y \rfloor = 1
    $$
    所以当$(2x+1) \mod 4=0$,也即$x=2n+\frac{3}{2},\ n\in Z $时:
    $$
        \lceil \frac{2x+1}{4} \rceil - \lfloor\frac{2x+1}{4} \rfloor = 0
    $$
    否则:
    $$
        \lceil \frac{2x+1}{4} \rceil - \lfloor\frac{2x+1}{4} \rfloor = 1
    $$
    可得:
    $$
        \begin{aligned}
            \lceil\frac{2 x+1}{2}\rceil-\lceil\frac{2 x+1}{4}\rceil+\lfloor\frac{2 x+1}{4}\rfloor & =  \lceil x+\frac{1}{2} \rceil - [x\neq 2n+\frac{3}{2}]                    \\
                                                                                                  & =\lceil x \rceil + [\{x\}>\frac{1}{2}] +[\{x\}=0] - [x\neq 2n+\frac{3}{2}] \\
                                                                                                  & =\left\{
            \begin{aligned}
                 & \lceil x \rceil,\ \{x\}> \frac{1}{2}                          \\
                 & \lceil x \rceil,\ \{x\}\le \frac{1}{2},\  (2x+1)\%4=0         \\
                 & \lfloor x \rfloor,\ 0<\{x\}\le \frac{1}{2},\  (2x+1)\%4\neq 0 \\
                 & \lfloor x \rfloor,\ \{x\}=0,\  (2x+1)\%4\neq 0                \\
            \end{aligned}
            \right.                                                                                                                                                            \\
        \end{aligned}
    $$

    \textbf{2.}

    设第n个元素为$a_n$,当$a_n=m$时,前面小于m的数$1,2,2,...,m-1$的个数是:
    $$
        \begin{aligned}
            \sum_{i=1}^{m-1}=\frac{m(m-1)}{2}
        \end{aligned}
    $$
    值为$m$的数组元素的个数是$m$,所以值为m的数组元素$a_n$的索引$n$的范围是:
    $$
        \begin{aligned}
            \frac{m(m-1)}{2} <  & n \le \frac{m(m-1)}{2}+m    \\
            m^2-m <             & 2n \le m^2+m                \\
            m^2-m +\frac{1}{4}< & 2n \le m^2+m +\frac{1}{4}   \\
            (m-\frac{1}{2})^2<  & 2n \le (m+\frac{1}{2})^2    \\
            m-\frac{1}{2}<      & \sqrt{2n} \le m+\frac{1}{2} \\
        \end{aligned}
    $$
    可得m的范围:
    \begin{equation}
        \label{equation:mRange}
        \sqrt{2n}-\frac{1}{2}\le      m  < \sqrt{2n}+\frac{1}{2} \\
    \end{equation}

    假设$\sqrt{2n}-\frac{1}{2}$是整数,即$\sqrt{2n}-\frac{1}{2}=k \in Z$:
    $$
        \begin{aligned}
            2n=(k+\frac{1}{2})^2  \\
            2n=k^2+k+\frac{1}{4}  \\
            2n-k(k+1)=\frac{1}{4} \\
        \end{aligned}
    $$
    等式左边是整数,右边不是整数,显然推出矛盾,故$\sqrt{2n}-\frac{1}{2}$不是整数,那么$\sqrt{2n}-\frac{1}{2}+1=\sqrt{2n}+\frac{1}{2}$也不是整数,所以:
    $$
        \begin{aligned}
            \left\lceil\sqrt{2n}-\frac{1}{2}\right\rceil = \left\lfloor\sqrt{2n}-\frac{1}{2}+1\right\rfloor =\left\lfloor\sqrt{2n}+\frac{1}{2}\right\rfloor \\
        \end{aligned}
    $$
    再带入不等式(\ref{equation:mRange})可得:
    $$
        \begin{aligned}
            \sqrt{2n}-\frac{1}{2}\le  m  \le \left\lfloor\sqrt{2n}+\frac{1}{2}\right\rfloor                           \\
            \left\lceil\sqrt{2n}-\frac{1}{2}\right\rceil \le  m  \le \left\lfloor\sqrt{2n}+\frac{1}{2}\right\rfloor   \\
            \left\lfloor\sqrt{2n}+\frac{1}{2}\right\rfloor \le  m  \le \left\lfloor\sqrt{2n}+\frac{1}{2}\right\rfloor \\
        \end{aligned}
    $$
    故:
    $$a_n = m  = \left\lfloor\sqrt{2n}+\frac{1}{2}\right\rfloor $$


\end{homeworkProblem}


\pagebreak


\begin{homeworkProblem}

    求解递归式:
    \begin{equation}
        \label{equation:recursiveFormula}
        \begin{cases}
            a_0=1; \\
            a_n=a_{n-1}+\left\lfloor\sqrt{a_{n-1}}\right\rfloor,\ n>0.
        \end{cases}
    \end{equation}

    \solution

    当$a_n=m^2$时:
    \begin{equation}
        \begin{aligned}
             & a_{n+1} = m^2+m,\                   & a_{n+2} = m^2+m+m=m^2+2m        \\
             & a_{n+3} = m^2+2m+m=(m+1)^2+(m-1),\  & a_{n+4} = m^2+3m+m+1=(m+1)^2+2m \\
             & a_{n+5} = m^2+5m+2=(m+2)^2+(m-2),\  & a_{n+6} = m^2+3m+m+1=(m+2)^2+2m \\
        \end{aligned}
    \end{equation}
    引理:
    \begin{equation}
        \label{equation:guess}
        a_{n+2k+1}=(m+k)^2+(m-k),\  a_{n+2k+2}=(m+k)^2+2m,\ 0\le k \le m \\
    \end{equation}
    用数学归纳法证明引理(\ref{equation:guess}):
    \begin{itemize}
        \item 当$k=0$时,$a_{n+2k+1}=a_{n+1}=m^2+m$,$a_{n+2k+2}=a_{n+2}=m^2+2m$,引理(\ref{equation:guess})成立
        \item 假设$k=l, \ l<m$时引理(\ref{equation:guess})成立,当$k=l+1$时:
              $$
                  \begin{aligned}
                       & a_{n+2(l+1)+1}=a_{n+2l+3}=(m+l)^2+2m+m+l=(m+(l+1))^2+m-(l+1)           \\
                       & a_{n+2(l+1)+2}=a_{n+2l+4}=(m+(l+1))^2+m-(l+1) + m+(l+1)=(m+(l+1))^2+2m \\
                  \end{aligned}
              $$
    \end{itemize}
    综上引理(\ref{equation:guess})成立。令$k=m$,可得:
    $$
        a_{n+2m+1}=(m+m)^2+(m-m)=(2m)^2
    $$
    此时又得到一个完全平方数。这样一来,我们可以发现,以某一完全平方数$a_n=m^2$为起点,可以利用引理(\ref{equation:guess})计算出其后面的$2m+1$个数,即:$a_n,a_{n+1},..,a_{n+2m+1}$,
    此时引理(\ref{equation:guess})已不能继续递推,需要重新更换起点,该起点得是完全平方数,而$a_{n+2m+1}=2m^2$刚好是完全平方数,以其为起点,又可以继续下一轮递推。

    由于$a_0=1$是完全平方数,因此,可以以$a_0=1$为起点递推,得到:$a_0,a_1,a_2,a_3$;
    再以$a_3$为起点,继续递推,得到:$a_3,a_4,a_5,..,a_8$;
    再以$a_8$为起点,继续递推,得到:$a_8,a_9,a_10,..,a_{17}$;
    ......;

    除去左起始端点,每轮递推依次得到$2^1+1,2^2+1,2^3+1,...,2^m+1,...$个数。

    左起始端点依次为$a_{2^1+1-3},a_{2^2+2-3},a_{2^3+3-3},..,a_{2^{m}+m-3},...$,
    所以,当$2^l+l-3\le n < 2^{l+1}+l-2$时:
    $$
        a_{n}=2^l+\left\lfloor\left(\frac{n-l-1}{2}\right)^2\right\rfloor
    $$

\end{homeworkProblem}

\end{document}
