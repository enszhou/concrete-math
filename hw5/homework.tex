\documentclass[fontset=windows]{article}

\usepackage{fancyhdr}
\usepackage{extramarks}
\usepackage{amsmath}
\usepackage{amsthm}
\usepackage{amsfonts}
\usepackage{tikz}
\usepackage{algorithm}
\usepackage{algpseudocode}
\usepackage{ctex}
\usepackage{manfnt}
\usepackage{xcolor}
\usepackage{pagecolor}

% \pagecolor{black}
% \color{white}


% hypergeometric function
\newmuskip\hfmuskip
\newcommand*\hf[4][6]{%
  \begingroup % only local assignments
  \hfmuskip=#1mu\relax
  \mathchardef\normalcomma=\mathcode`,
  % make the comma math active
  \mathcode`\,=\string"8000
  % and define it to be \hfcomma
  \begingroup\lccode`\~=`\,
  \lowercase{\endgroup\let~}\hfcomma
  % typeset the formula
%   {}_{#2}F_{#3}{\left(\genfrac..{0pt}{}{#4}{#5}\bigg|#6\right)}%
  F{\left(\genfrac..{0pt}{}{#2}{#3}\bigg|#4\right)}%
  \endgroup
}
\newcommand{\hfcomma}{{\normalcomma}\mskip\hfmuskip}
% hypergeometric function


\renewcommand{\algorithmicrequire}{\textbf{Input:}}  % Use Input in the format of Algorithm
\renewcommand{\algorithmicensure}{\textbf{Output:}} % Use Output in the format of Algorithm

\usetikzlibrary{automata,positioning}

%
% Basic Document Settings
%

\topmargin=-0.45in
\evensidemargin=0in
\oddsidemargin=0in
\textwidth=6.5in
\textheight=9.0in
\headsep=0.25in

\linespread{1.1}

\pagestyle{fancy}
\lhead{\hmwkAuthorName}
\chead{\hmwkClass: \hmwkTitle}
\rhead{\firstxmark}
\lfoot{\lastxmark}
\cfoot{\thepage}

\renewcommand\headrulewidth{0.4pt}
\renewcommand\footrulewidth{0.4pt}

\setlength\parindent{0pt}

%
% Create Problem Sections
%

\newcommand{\enterProblemHeader}[1]{
    \nobreak\extramarks{}{Problem \arabic{#1} continued on next page\ldots}\nobreak{}
    \nobreak\extramarks{Problem \arabic{#1} (continued)}{Problem \arabic{#1} continued on next page\ldots}\nobreak{}
}

\newcommand{\exitProblemHeader}[1]{
    \nobreak\extramarks{Problem \arabic{#1} (continued)}{Problem \arabic{#1} continued on next page\ldots}\nobreak{}
    \stepcounter{#1}
    \nobreak\extramarks{Problem \arabic{#1}}{}\nobreak{}
}

\setcounter{secnumdepth}{0}
\newcounter{partCounter}
\newcounter{homeworkProblemCounter}
\setcounter{homeworkProblemCounter}{1}
\nobreak\extramarks{Problem \arabic{homeworkProblemCounter}}{}\nobreak{}

%
% Homework Problem Environment
%
% This environment takes an optional argument. When given, it will adjust the
% problem counter. This is useful for when the problems given for your
% assignment aren't sequential. See the last 3 problems of this template for an
% example.
%
\newenvironment{homeworkProblem}[1][-1]{
    \ifnum#1>0
        \setcounter{homeworkProblemCounter}{#1}
    \fi
    \section{Problem \arabic{homeworkProblemCounter}}
    \setcounter{partCounter}{1}
    \enterProblemHeader{homeworkProblemCounter}
}{
    \exitProblemHeader{homeworkProblemCounter}
}

\newcommand{\hmwkTitle}{Homework 4}
\newcommand{\hmwkDueDate}{May 18, 2022}
\newcommand{\hmwkClass}{Concrete Math}
\newcommand{\hmwkClassInstructor}{Professor Xie Wei}
\newcommand{\hmwkStuNum}{\textbf{SA21011018}}
\newcommand{\hmwkAuthorName}{\textbf{Zhou Enshuai}}

%
% Title Page
%

\title{
    \vspace{2in}
    \textmd{\textbf{\hmwkClass:\ \hmwkTitle}}\\
    \normalsize\vspace{0.1in}\small{Due\ on\ \hmwkDueDate\ at 14:00}\\
    \vspace{0.1in}\large{\textit{\hmwkClassInstructor}}
    \vspace{3in}
}

\author{\hmwkStuNum\\ \hmwkAuthorName}
\date{\today}

\renewcommand{\part}[1]{\textbf{\large Part \Alph{partCounter}}\stepcounter{partCounter}\\}

%
% Various Helper Commands
%

% Useful for algorithms
\newcommand{\alg}[1]{\textsc{\bfseries \footnotesize #1}}

% For derivatives
\newcommand{\deriv}[1]{\frac{\mathrm{d}}{\mathrm{d}x} (#1)}

% For partial derivatives
\newcommand{\pderiv}[2]{\frac{\partial}{\partial #1} (#2)}

% Integral dx
\newcommand{\dx}{\mathrm{d}x}

% Alias for the Solution section header
\newcommand{\solution}{~\\ \textbf{\Large Solution}}

% Probability commands: Expectation, Variance, Covariance, Bias
\newcommand{\E}{\mathrm{E}}
\newcommand{\Var}{\mathrm{Var}}
\newcommand{\Cov}{\mathrm{Cov}}
\newcommand{\Bias}{\mathrm{Bias}}

\begin{document}

\maketitle

\pagebreak

\begin{homeworkProblem}

    Show that $2\sqrt{n+1}-2\le \sum_{k=1}^{n} \frac{1}{\sqrt{k}} \le 2\sqrt{n}-1$.

    \solution

    ~\\
    用积分放缩法证明上述不等式,采用的积分函数是$f(x) = \frac{1}{\sqrt{x}}$:
    \begin{align*}
        \int_{1}^{n+1} \frac{1}{\sqrt{x}} dx \le & \sum_{k=1}^{n}\frac{1}{\sqrt{k}} \le  1 + \int_{1}^{n} \frac{1}{\sqrt{x}} dx \\
        2\sqrt{x} \Big|_{1}^{n+1} \le            & \sum_{k=1}^{n}\frac{1}{\sqrt{k}} \le  1 +  2\sqrt{x} \Big|_{1}^{n}           \\
        2\sqrt{n+1}-2\le                         & \sum_{k=1}^{n} \frac{1}{\sqrt{k}} \le 2\sqrt{n}-1
    \end{align*}


\end{homeworkProblem}

\pagebreak

\begin{homeworkProblem}

    Using the idea of generating function, solve the recurrences:

    \begin{enumerate}
        \item  $f_0 = 1,f_1 = 2, f_n= 2f_{n-1} - f_{n-2} + (-2)^n\ for\ n \ge2$.
        \item $ g_0 = 0,h_0 = 1,g_1 = h_1 = 2, g_n=2h_{n-1}-g_{n-2}, h_n= g_{n-1} -h_{n-2}\ for\ n\ge2$.
    \end{enumerate}

    \solution

    ~\\
    1.

    注意到这里$f_1=2$,不满足初始递推关系,修正如下:
    \begin{align*}
        f_n= 2f_{n-1} - f_{n-2} + (-2)^n[n\ge 0] + 2[n=1]
    \end{align*}
    带入生成函数可得:
    \begin{align*}
        F(z) & = \sum_n f_n z^n = \sum_n\left(2f_{n-1} - f_{n-2} + (-2)^n[n\ge 0] + 2[n=1]\right)z^n \\
             & =\sum_n2f_{n-1}z^n - \sum_nf_{n-2}z^n + \sum_n(-2)^n[n\ge 0]z^n + \sum_n2[n=1]z^n     \\
             & =2zF(z) - z^2F(z) + \frac{1}{1+2z} + 2z                                               \\
    \end{align*}
    解得:
    \begin{align*}
        F(z) & =  \frac{1}{(1+2z)(z-1)^2} + \frac{2z}{(z-1)^2}                                                       \\
             & = \frac{4}{9}\cdot\frac{1}{1+2z} - \frac{16}{9}\cdot\frac{1}{1-z} + \frac{7}{3}\cdot\frac{1}{(1-z)^2}
    \end{align*}
    所以:
    \begin{align*}
        f_n & = [z^n]F(z) =  \frac{4}{9}(-2)^n- \frac{16}{9} + \frac{7}{3}(n+1),\ for\ n \ge 0.
    \end{align*}

    2.

    注意到这里$h_0=1,h_1=2$,不满足初始递推关系,对所有的$n$修正如下:
    \begin{align*}
        \left\{
        \begin{aligned}
             & h_n = g_{n-1}-h_{n-2}+[n=0]+2[n=1] \\
             & g_n=2h_{n-1}-g_{n-2}
        \end{aligned}
        \right.
    \end{align*}
    从而
    \begin{align*}
        \left\{
        \begin{aligned}
             & H(z)=\sum_n h_n z^n = zG(z)-z^2H(z)+1+2z \\
             & G(z)=\sum_n g_n z^n = 2zH(z)-z^2G(z)
        \end{aligned}
        \right.
    \end{align*}
    解得:
    \begin{align*}
        \left\{
        \begin{aligned}
             & H(z)=\frac{(1+z^2)(1+2z)}{z^4+1} = (2z^3+z^2+2z+1)\frac{1}{z^4+1}= (2z^3+z^2+2z+1)\sum_{n\ge0} (-1)^nz^{4n} \\
             & G(z)=\frac{2z(1+2z)}{z^4+1} = (4z^2+2z)\frac{1}{z^4+1} = (4z^2+2z)\sum_{n\ge0} (-1)^nz^{4n}                 \\
        \end{aligned}
        \right.
    \end{align*}
    所以:
    \begin{align*}
        h_n = [z^n]H(z)=\left\{
        \begin{aligned}
             & (-1)^\frac{n}{4},\ n\%4=0    \\
             & 2(-1)^\frac{n-1}{4},\ n\%4=1 \\
             & (-1)^\frac{n-2}{4},\ n\%4=2  \\
             & 2(-1)^\frac{n-3}{4},\ n\%4=3
        \end{aligned}
        \right.
        =\left\{
        \begin{aligned}
             & (-1)^{\lfloor\frac{n}{4}\rfloor},\ n\%4=0 \ \mathrm{or}\ n\%4=2       \\
             & 2\cdot(-1)^{\lfloor\frac{n}{4}\rfloor},\ n\%4=1 \ \mathrm{or}\ n\%4=3
        \end{aligned}
        \right.
    \end{align*}
    \begin{align*}
        g_n = [z^n]G(z)=\left\{
        \begin{aligned}
             & 0,\ n\%4=0\ \mathrm{or}\ n\%4=3 \\
             & 2(-1)^\frac{n-1}{4},\ n\%4=1    \\
             & 4(-1)^\frac{n-2}{4},\ n\%4=2
        \end{aligned}
        \right.
        =\left\{
        \begin{aligned}
             & 0,\ n\%4=0\ \mathrm{or}\ n\%4=3                 \\
             & 2\cdot(-1)^{\lfloor\frac{n}{4}\rfloor},\ n\%4=1 \\
             & 4\cdot(-1)^{\lfloor\frac{n}{4}\rfloor},\ n\%4=2
        \end{aligned}
        \right.
    \end{align*}

\end{homeworkProblem}

\pagebreak

\begin{homeworkProblem}

    A random variable $X$ is said to have the Poisson distribution with mean
    $\lambda$ if $\mathrm{Pr}(X=k) = \frac{e^{-\lambda}\lambda^k}{k!}$ for all $k\in \mathbb{N} $.
    Let $X_1$ and $X_2$ be independent random
    Poisson variables both with variance $t$.
    Calculate the distribution of $X_1 + X_2$ .

    \solution

    先证明服从泊松分布的均值为$\lambda$的随机变量X的方差也是$\lambda$,设$X\sim Pois(\lambda)$:
    \begin{align*}
        VarX & = EX^2-(EX)^2 = EX^2 - \lambda^2                                                                                 \\
             & = \sum_{k\ge0}\frac{e^{-\lambda}\lambda^k k^2}{k!} - \lambda^2                                                   \\
             & = e^{-\lambda}\lambda\sum_{k\ge1}\frac{\lambda^{k-1} k}{(k-1)!} - \lambda^2                                      \\
             & = e^{-\lambda}\lambda\sum_{k\ge0}\frac{\lambda^{k} (k+1)}{k!} - \lambda^2                                        \\
             & = e^{-\lambda}\lambda\sum_{k\ge0}\frac{\frac{\mathrm{d}\lambda^{k+1}}{\mathrm{d}\lambda}}{k!} - \lambda^2        \\
             & = e^{-\lambda}\lambda\frac{\mathrm{d}(\sum_{k\ge0}\frac{\lambda^{k+1}}{k!})}{\mathrm{d}\lambda} - \lambda^2      \\
             & = e^{-\lambda}\lambda\frac{\mathrm{d}(\lambda\sum_{k\ge0}\frac{\lambda^{k}}{k!})}{\mathrm{d}\lambda} - \lambda^2 \\
             & = e^{-\lambda}\lambda\frac{\mathrm{d}(\lambda e^\lambda)}{\mathrm{d}\lambda} - \lambda^2                         \\
             & = e^{-\lambda}\lambda(e^\lambda+\lambda e^\lambda) - \lambda^2                                                   \\
             & = \lambda                                                                                                        \\
    \end{align*}

    所以$X_1$ 和 $X_2$独立同分布于参数为t的泊松分布,所以$X_1+X_2$也服从泊松分布,即$X_1\sim Pois(t)$和$X_2\sim Pois(t)$,
    设$X = X_1+X_2$,则X的概率分布为:
    \begin{align*}
        \mathrm{Pr}(X=k) & = \mathrm{Pr}(X_1+X_2=k)                                                          \\
                         & = \sum_{i}\mathrm{Pr}(X_1=i,X_2=k-i)                                              \\
                         & = \sum_{i}\left[\mathrm{Pr}(X_1=i)\cdot \mathrm{Pr}(X_2=k-i)\right]               \\
                         & = \sum_{i=0}^k\left[\frac{e^{-t}t^i}{i!}\cdot \frac{e^{-t}t^{k-i}}{(k-i)!}\right] \\
                         & = \frac{e^{-2t}t^{k}}{k!}\sum_{i=0}^k\frac{k!}{i!(k-i)!}                          \\
                         & = \frac{e^{-2t}t^{k}}{k!}\sum_{i=0}^k {k \choose i}                               \\
                         & = \frac{e^{-2t}t^{k}}{k!}\cdot2^k                                                 \\
                         & = \frac{e^{-2t}(2t)^{k}}{k!}                                                      \\
    \end{align*}
    即$X_1+X_2 \sim Pois(2t)$.

    事实上利用指数生成函数的二项卷积公式可以得到:若$X_1,X_2$独立,且$X_1\sim Pois(t_1),\ X_2\sim Pois(t_2)$,则$X_1+X_2 \sim Pois(t_1+t_2)$.
\end{homeworkProblem}

\pagebreak

\begin{homeworkProblem}
    If we toss a coin, it comes up heads with probability $p$, which is fixed
    but unknown. We toss the coin $n$ times (different tosses are independent), and give
    an estimate $\hat{p}$ of $p$. Given small $\varepsilon,\delta>0$,
    choose an $n$ as small as possible such that
    $\mathrm{Pr}(|\hat{p}-p| \ge \varepsilon)\le \delta$ is satisfied.

    \solution

    该实验是n次重复伯努利试验,设第$i$次抛掷硬币结果随机变量为$X_i$
    \begin{align*}
        X_i & = \begin{cases}
            1, & \text{if comes up head at the ith time} \\
            0, & \text{if comes up tail at the ith time}
        \end{cases},
        \quad 0\le i \le n
    \end{align*}
    所以$\mathrm{Pr}(X_i=1) = p$,则$EX_i=p$,且$X_1,X_2,\cdots,X_n$相互独立.

    设$X=\sum_{i=1}^nX_i$,则$EX=\sum_{i=1}^nEX_i=np=\mu$,

    本题使用了下面的chernoff bound:
    \begin{align}
        \mathrm{Pr}(|X-\mu|\ge \eta \mu)\le  2e^{-\eta ^2\mu/3},\quad 0\le \eta \le 1
        \label{chernoff}
    \end{align}

    n次重复实验对$p$的估计为随机变量$\hat{p}$
    \begin{align*}
        \hat{p} = \frac{1}{n}\sum_{i=1}^nX_i = \frac{X}{n}
    \end{align*}
    所以
    \begin{align*}
        \mathrm{Pr}(|\hat{p}-p| \ge \varepsilon)
         & = \mathrm{Pr}\left(\left|\frac{X}{n}-p\right| \ge \varepsilon\right)
        = \mathrm{Pr}\left(\left|X-np\right| \ge n\varepsilon\right)                \\
         & = \mathrm{Pr}\left(\left|X-\mu\right| \ge \frac{np\varepsilon}{p}\right)
        = \mathrm{Pr}\left(\left|X-\mu\right| \ge \mu\frac{\varepsilon}{p}\right)
    \end{align*}
    因为$p$是定值,而$\varepsilon$足够小,所以$ 0\le \frac{\varepsilon}{p} \le 1$,
    所以取不等式(\ref{chernoff})中的$\eta=\frac{\varepsilon}{p}$可得:
    \begin{align*}
        \mathrm{Pr}(|X-\mu|\ge \frac{\varepsilon}{p} \mu)
        \le  2e^{-(\frac{\varepsilon}{p})^2\mu/3}
        = 2e^{-n\varepsilon^2/3p}
    \end{align*}
    所以若要使$\mathrm{Pr}(|\hat{p}-p| \ge \varepsilon)=\mathrm{Pr}\left(\left|X-\mu\right| \ge \mu\frac{\varepsilon}{p}\right)\le \delta$成立,
    只需要让$2e^{-n\varepsilon^2/3p} \le \delta$成立即可:
    \begin{align*}
        2e^{-n\varepsilon^2/3p} \le \delta
         & \Leftrightarrow -n\varepsilon^2/3p \le \ln{\frac{\delta}{2}}         \\
         & \Leftrightarrow n\varepsilon^2/3p \ge \ln{\frac{2}{\delta}}          \\
         & \Leftrightarrow  n \ge \frac{3p}{\varepsilon^2}\ln{\frac{2}{\delta}} \\
    \end{align*}
    所以一个尽可能小的满足条件的$n$应该是
    $$n=\left\lceil\frac{3p}{\varepsilon^2}\ln{\frac{2}{\delta}}\right\rceil$$

\end{homeworkProblem}

\pagebreak


\end{document}
