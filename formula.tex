\documentclass[fontset=windows]{article}

\usepackage{fancyhdr}
\usepackage{extramarks}
\usepackage{amsmath}
\usepackage{amsthm}
\usepackage{amsfonts}
\usepackage{tikz}
\usepackage{algorithm}
\usepackage{algpseudocode}
\usepackage{ctex}
\usepackage{manfnt}
\usepackage{xcolor}
\usepackage{pagecolor}


\begin{document}

\begin{align}
      & EX = G_X^\prime (1)                                         \\
      & VarX=G_X^{\prime\prime} (1)+G_X^\prime (1)-G_X^\prime (1)^2 \\
      & \left\{
     \begin{aligned}
           & \kappa_1=\mu_1=3                      \\
           & \kappa_2=\mu_2-\mu_1^2=0              \\
           & \kappa_3=\mu_3-3\mu_1\mu_2+2\mu_1^3=0 \\
     \end{aligned}
     \right.                                                        \\
     \mathrm(coin)
      & \left\{
     \begin{aligned}
           & EX = \sum_{k=1}^m \widetilde A_{(k)}[A_{(k)}=A^{(k)}]               \\
           & VX = (EX)^2-\sum_{k=1}^m (2k-1) \widetilde A_{(k)}[A_{(k)}=A^{(k)}]
     \end{aligned}
     \right.                                                        \\
\end{align}

欧拉求和公式:
\begin{align}
     \begin{gathered}
          \sum_{a \le k<b} f(k)=\int_{a}^{b} f(x) \mathrm{d} x-\left.\frac{1}{2} f(x)\right|_{a} ^{b}+\left.\sum_{k=1}^{m} \frac{B_{2 k}}{(2 k) !} f^{(2 k-1)}(x)\right|_{a} ^{b} \\
          +O\left((2 \pi)^{-2 m}\right) \int_{a}^{b}\left|f^{(2 m)}(x)\right| \mathrm{d} x
     \end{gathered}
\end{align}

伯努利数:
\begin{align}
     \begin{array}{c|ccccccccccccc}
          n            & 0 & 1            & 2           & 3 & 4             & 5 & 6            & 7 & 8             & 9 & 10           & 11 & 12                \\
          \hline B_{n} & 1 & \frac{-1}{2} & \frac{1}{6} & 0 & \frac{-1}{30} & 0 & \frac{1}{42} & 0 & \frac{-1}{30} & 0 & \frac{5}{66} & 0  & \frac{-691}{2730}
     \end{array}
\end{align}

对数形式斯特林近似
\begin{align}
     \ln n !=n \ln n-n+\frac{\ln n}{2}+\frac{\ln 2\pi}{2}+\frac{1}{12 n}-\frac{1}{360 n^{3}}+\frac{\varphi_{2, n}}{1260 n^{5}}
\end{align}

钟形求和
\begin{align}
     \Theta_{n}=\sum_{k} e^{-k^2/n} = \sqrt{\pi n} + O(n^{-M})
\end{align}

泰勒展开
\begin{align*}
     f(x) = \sum_{n=0}^{\infty}\frac{f^{(n)}(a)}{n!}(x-a)^n
\end{align*}

\end{document}